\clearpage

\section{Definitions}

In order to prevent misunderstandings and to clarify the subject of the thesis we begin from introducing
basic terms that we use in the reminder of this thesis.

\paragraph{Failure (crash)}
is a permanent lack of activity from a program. It may be caused by~programming error, lack of electricity etc.
We do not consider byzantine failures, i.e.~a~process may not misbehave in any way.

\paragraph{Catastrophic failure} is a failure of all processes at the same moment.

\paragraph{The failure model}
defines what type of failures can be tolerated by a system
\begin{tightList}[ \setlength{\leftmargin}{2\leftmargin}]
 \item[\textbf{Crash-Stop}] means that if a process failed, it failed permanently and must never be up again
 \item[\textbf{Crash-Recovery}] assumes that a crashed process may recover (i.e.\ start working again)
\end{tightList}

\paragraph{Correct process} is a process that did not crash, i.e.\ that works according to the expectations.

\paragraph{Consensus}
is a problem in distributed computing that encapsulates the task of reaching distributed agreement in the group of processes in the presence of faults.
The consensus protocol guarantees:

\begin{tightList}[\setlength{\leftmargin}{2\leftmargin}]
    \item[\textbf{Validity}] any value decided is a value proposed by some process,
    \item[\textbf{Agreement}] no two processes decide differently,
    \item[\textbf{Termination}] every correct process eventually decides,
    \item[\textbf{Integrity}] no process decides twice.
\end{tightList}

\paragraph{Instance (ballot)} is a single logical run of an algorithm. In order to decide on multiple values, many consecutive \textit{consensus instances} are executed, each identified by its \textit{ID}.

\paragraph{Value} is the value consensus agrees upon. In JPaxos, the value always consists of the client requests.

\paragraph{State machine}
is either a program, an algorithm or a protocol that can be described by its state and that can transit to other state only by receiving a command.
There are no restrictions how the state may change.

\paragraph{Deterministic state machine}
is a state machine that from the same state under the same command will always change state in the same way.
Thanks to this property, one may describe the state machine's state by the initial state and consecutive commands. 

\paragraph{State machine replication} is a method for implementing a replicated state machine, i.e.\ for maintaining multiple copies of identical state machine, possibly on separate network nodes, usually in order to gain availability of the state machine and durability of its state. State machine replication includes proper client handling and guaranteeing uniform state changes in all replicas.

\paragraph{Service}
is a program that receives requests (or commands) and executes them generating a response.

\paragraph{Deterministic service}
is a service that in a given state will always given the same response for a given command, and will always change its state to the same state.

\paragraph{Snapshot}
is the saved state of a service at a specified point in time that can be used by~the same service to restore its state to the exact state from the specified point in time. The snapshot should be in format that may be easily stored and transferred via network.

\paragraph{Client}
is the program sending requests to the service.

\paragraph{Atomic (total order) broadcast}
is a networking primitive providing send-to-all communication for which holds:
\begin{tightList}[\setlength{\leftmargin}{2\leftmargin}]
 \item[\textbf{Validity}] if a process delivers a message, it was broadcast by some process,
 \item[\textbf{Agreement}] if a process delivers a message, all valid processes will deliver it,
 \item[\textbf{Integrity}] a message is delivered only if it was broadcasted previously, and it reaches all valid processes at most once,
 \item[\textbf{Total order}] each two messages are delivered in the same order at every process.
\end{tightList}

\noindent The atomic broadcast problem is equivalent to consensus, i.e.\ if one can be solved, then the other also can be solved.

\paragraph{Stable storage}
is the memory that survives crashes. Usually stable storage denotes a~hard drive.
Sometimes also \textbf{volatile storage} name is used, to name memory that does not survive crashes, like the RAM memory.

\noindent The writes to the stable storage must be permanent contrary to the volatile memory. Thus if a hard drive is used, the writes must be synchronous.

\section{Theoretical limitations}

\paragraphNewline{The FLP impossibility result}

The consensus problem is not solvable in an asynchronous system where at least one process may crash and processes communicate by sending messages. This fact has been proved in the \cite{FLP}.

\paragraphNewline{Number of messages}

In the best case no algorithm is able to solve consensus faster than in the time needed to send one message with the value and one message not carrying value agreed upon.

Our implementation, as described in section \ref{par:bestCaseMessages}, is able to decide client commands in such time. Moreover, assuming that no network congestion occurs and no messages get lost the average time is equal to the best-case time.

However, with TCP and simple UDP it is not possible to use either multicast or~broadcast primitive; this prolongs the real time needed for sending a message, as~in the JPaxos network module \emph{broadcast} is translated to $n$ identical \textit{unicast} messages.

Other Paxos implementations do use a low-level multicast protocol for reducing communication. As an example, the RingPaxos \cite{Mar10} uses IP-multicast. The most significant gain from using multicast is scalability -- without the low-level multicast, each additional replica decreases the overall performance.

