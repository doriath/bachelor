\chapter{Introduction}

With the increase of usage and importance of Internet services in everyday life, there also increases demand on service reliability and constant availability.
Providing these features to the services is a complex matter, involving a trade-off between user satisfaction and service resource demands.

The most common and effective method of providing reliable services is replication. Applying the replication requires only minor changes for non-replicated service to become replicated one. Thus, this method is usable both for developing new services using well known design patterns as well as transforming existing services into replicated ones.

Each replication approach relies on some group communication mechanism, usually providing strong guarantees. For example, the total-order (atomic) broadcast is a very useful for interconnecting replicas -- it simplifies the design and implementation of the replication protocols, as it allows for dividing the system into two main modules with clearly visible responsibilities.

We have chosen to implement one of the replication approaches -- one that we think is very promising.
It is called active replication. In this approach, each replica does the same work. The replicas must behave identically and therefore we require the service to behave deterministically.

Every library for replicating services should address its main purpose, i.e.\ to guarantee high availability of service as well as durability of the service work results.


%\begin{minipageWithIndent}{0.9\textwidth}
%{
%\bfseries
%
%The goal of this bachelor's thesis is to implement JPaxos -- Java library and runtime system for efficient service replication.
%
%JPaxos requires deterministic behaviour from the service, does not accept replica misbehaviour and works with static group of replicas.
%
%JPaxos allows for replica crashes as well as recovering the crashed replicas, ensures the durability in spite of the failures, including failure of all the replicas at once. It provides the availability of service if the majority of replicas are correct.
%}
%\end{minipageWithIndent}
{\bfseries
As part of our diploma project, we have developed a library and runtime system for efficient service replication. Our system, written in Java, implements the Paxos protocol \cite{Lam98} for distributed consensus.

JPaxos requires deterministic behaviour from the service, does not tolerate Byzantine failures and works with static group of replicas.

JPaxos supports replica crashes as well as recovering the crashed replicas, \linebreak ensures durability in spite of failures, including the failure of all replicas at once. \linebreak It provides availability of service as long as a majority of replicas are correct.
}

\section{Thesis organization}

In the opening chapter we shortly characterize articles concerning Paxos algorithm and present the most significant definitions as well as theoretical results concerning Paxos.
The second chapter provides description of the Paxos algorithm.
In chapter 3 we describe the design of JPaxos -- its architecture, modules, important data structures and threading model.
Chapter 4 is devoted to the implementation work related to the state machine replication, including snapshotting, catch-up and achieving transparency for the service.
Fifth chapter contains description of optimizations we used in order to achieve better performance of the library.
Chapter 6 describes recovery related issues, both for Paxos and for the service replication.
The last chapter concludes the thesis.

\section{Related work}

The Paxos and MultiPaxos algorithms have been proposed by L. Lamport in \textit{The Part-Time Parliament}~\cite{Lam98}. Since the article presented the algorithm in quite specific form, the Paxos protocol has been described again by Lamport in \textit{Paxos Made Simple}~\cite{Lam01}.

Since then the Paxos protocol has been described from the theoretical point of view. Improvements and their proofs of correctness were presented as well as behaviour with byzantine crashes has been characterized.

One of the first real-world Paxos application has been described in 2007 (\textit{Paxos made live}~\cite{CGR07}). The article describes certain implementation issues concerning the Paxos algorithm in the Chubby distributed filesystem developed at Google.

A year later, in \textit{Paxos for system builders}~\cite{AK08}, the Paxos protocol has been described from programmer point of view.

In 2009, and subsequently in 2010, M. Primi et al.\ released two implementations of~Paxos designed to achieve good performance. Both of these implementations have been released as open-source. The first -- libPaxos$^2$ -- is described in Primi's master thesis, the latter -- RingPaxos -- is described in \textit{Ring Paxos: A High-Throughput Atomic Broadcast Protocol}~\cite{Mar10}.

The crash-recovery model has also been the subject of many research papers. \linebreak The \textit{Atomic Broadcast in Asynchronous Crash-Recovery Distributed Systems}~\cite{rodriguez2000atomic} provides definitions concerning recovery as well as shows a method to transform any crash-stop consensus protocol to the crash-recovery one.

A summary of the crash-recovery approaches for the Paxos algorithm can be found in N. Santos technical report~\cite{Nun10}. All algorithms implemented by us are described there.

\section{Contributions}

Jan Kończak has prepared the design and implementation~of:
\begin{tightList}
  \item[\textbullet] catch-up module (Section \ref{sec:catch_up}),
  \item[\textbullet] snapshotting module (Section \ref{sec:snapshotting}),
  \item[\textbullet] service proxy  (Section \ref{sec:serviceProxy}),
  \item[\textbullet] benchmark project.
\end{tightList}

\noindent Tomasz Żurkowski has prepared the design and implementation of:
\begin{tightList}
  \item[\textbullet] unit tests for JPaxos project,
  \item[\textbullet] client manager module, 
  \item[\textbullet] network module for communication between replicas,
  \item[\textbullet] view-based recovery (Section \ref{sec:view_ss}),
  \item[\textbullet] disc writers for full stable storage recovery.
\end{tightList}

\noindent JPaxos has been created in collaboration with Laboratoire de Systèmes Répartis at École Polytechnique Fédérale de Lausanne (Distributed Systems Laboratory at EPFL). Part of the ideas and implementations work has been done by Nuno Santos.

\noindent Nuno Santos has prepared the design and implementation~of:
\begin{tightList}
 \item[1.] Implementing, enhancing and fixing JPaxos
  \begin{tightList}
    \item[\textbullet] initial implementation 30.06.2009 (had about $13\%$ of current code size):
    \begin{tightList}
      \item[---] core MultiPaxos algorithm
      \item[---] failure detector and leader selection
      \item[---] basic replica, client and service implementation
    \end{tightList}
    \item[\textbullet] Renaming classes and cleaning up the code 22.09.2009
    \item[\textbullet] Changing Benchmark project 06.10.2009
    \item[\textbullet] Code cleanup 30.10.2009
    \item[\textbullet] Tampering with the code 19.01.2010
    \item[\textbullet] Improving the dispatcher thread 29.01.2010
    \item[\textbullet] Moving faliure detector from own thread to Paxos thread, tampering with the code 04.02.2020
    \item[\textbullet] Adding support for paxos.properties file instead of hardcoded settings 05.02.2020
    \item[\textbullet] Improving paxos.properties file 10.05.2010
    \item[\textbullet] Code cleanup, theoretical bug fix 18.05.2010
    \item[\textbullet] Moving catch-up from separate thread to the Paxos thread, rewrite proposer part batching, removing value from accept messages, improving retransmitter, rewriting TCP communication 22.06.2010
    \item[\textbullet] Introducing priority queue in the Dispatcher thread, tampering with the code 24.06.2010
    \item[\textbullet] Testing and fixes after branch merge 11.08.2010
    \item[\textbullet] Improved batching 31.08.2010
    \item[\textbullet] 'Improving' the TCP so that it does not work properly 30.09.2010
  \end{tightList}

  \item[2. ] Benchmarking and using JPaxos
  \begin{tightList}
    \item[\textbullet] Started Leader Oracle (not included in JPaxos) 22.09.2009
    \item[\textbullet] Changes to Leader Oracle 22.01.2010
    \item[\textbullet] Changes to Leader Oracle 04.02.2020 - 32.02.2010
    \item[\textbullet] Modifications for improving benchmarks 07,15,26.07.2010
    \item[\textbullet] Further improvements for benchmarks 17.08.2010
    \item[\textbullet] Improving benchmarks again... 13.09.2010
  \end{tightList}
\end{tightList}


\noindent The remaining modules were the result of our joint work.
