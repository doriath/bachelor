
\chapter{Introduction}

With the increase of usage and importance of Internet services in everyday life, there also increases demand on service reliability and constant availability.
Providing these features to the services is a complex matter, involving a trade-off between user satisfaction and service resource demands.

The most common and effective method of providing reliable services is replication. Applying the replication requires only minor changes for non-replicated service to become replicated one. Thus, this method is usable both for developing new services using well known design patterns as well as transforming existing services into replicated ones.

Each replication approach relies on some group communication mechanism, usually providing strong guarantees. For example the total-order (atomic) broadcast is a very useful for interconnecting replicas -- it simplifies the design and implementation of the replication protocols, as it allows for dividing the system into two main modules with clearly visible responsibilities.

The services vary, and so the approaches for replication must also vary. A library able to replicate every service would not only be hard to implement, but also could be too complicated for the user.

We have chosen to implement one of the replication approaches -- one that we think is very promising.
It is called active replication of services. In this approach each service replica does the same work. The replicas must behave identically, therefore we require the service to behave deterministically.

Every library for replicating services should address its main purpose, i.e. to guarantee high availability of service as well as durability of the service work results
 -- while having minimum requirements for the environment. % ← TODO

Our software design goal is to work well in real networks, % ← TODO
survive temporary network crashes and be able to recover from replica crashes as well.

{
\bfseries
The goal of this bachelor's thesis is to create Java library implementing state machine replication. The library should support the crash-recovery model of failures and should tolerate message loss and communication delays.
% TODO: ^^^^^^^^^^^^^^^
}

\section{Thesis organisation}
Second part of this chapter % TODO
shortly characterizes articles concerning Paxos algorithm and presents the most significant definitions as well as theoretical results concerning Paxos.

The second chapter provides description of Paxos algorithm used by us, including the modifications and implementation details.

In chapter 3 we describe the design of JPaxos -- its architecture, modules, important data structures and threading approach.

Implemented features, chapter 4, is devoted to the implementation work related to the state machine replication, including the snapshotting, catch-up and proper achieving transparency for the service.

Fifth chapter describes the recovery related issues, both for Paxos and for the state machine replication.

The last chapter concludes the thesis. %TODO: finish as this will be written

\section{Related work}



The Paxos and MultiPaxos algorithm have been proposed by L. Lamport in \textit{The Part-Time Parliament}~\cite{Lam98}. Since the article presented the algorithm in quite specific form, the Paxos protocol has been described again by Lamport in \textit{Paxos Made Simple}~\cite{Lam01}.

Since then the Paxos protocol has been described from the theoretical point of view. Improvements and their proofs of correctness were presented as well as behaviour with byzantine crashes has been characterised.

Some of the first real-world Paxos applications have been described in 2006 and 2007 (\textit{Paxos made\linebreak live}~\cite{CGR07}). The articles describe certain implementation issues concerning the Paxos algorithm in the Chubby distributed filesystem developed at Google.

A year later, in \textit{Paxos for system builders}~\cite{AK08}, the Paxos protocol has been described from programmer point of view.

In 2009, and subsequently in 2010, M. Primi et al. released two implementations of Paxos designed to achieve good performance. Both of these implementations have been released as open-source. The first -- libPaxos$^2$ -- is described in Primi's master thesis, the latter -- RingPaxos -- is described in \textit{Ring Paxos: A High-Throughput Atomic Broadcast Protocol}~\cite{Mar10}.

The crash-recovery model has also been the subject of many research papers. The \textit{Atomic Broadcast in Asynchronous Crash-Recovery Distributed Systems}~\cite{rodriguez2000atomic} provides definitions concerning recovery as well as shows a method to transform any crash-stop consensus protocol to the crash-recovery one.

A summary of the crash-recovery approaches for the Paxos algorithm can be found in N. Santos technical report~\cite{Nun10}. All algorithms implemented by us are described there.

\section{Contributions}
% TODO:  ^^^^^^^^^^^^^

Jan Kończak for the purpose of this thesis has prepared the design and implementation of:
\begin{tightList}
  \item[\textbullet] catch-up module (Section \ref{sec:catch_up}),
  \item[\textbullet] snapshotting module (Section \ref{sec:snapshotting}),
  \item[\textbullet] service proxy  (Section \ref{sec:serviceProxy}),
  \item[\textbullet] benchmark project.
\end{tightList}

\noindent Tomasz Żurkowski has prepared the design and implementation of:
\begin{tightList}
  \item[\textbullet] unit tests for JPaxos project,
  \item[\textbullet] client manager module, 
  \item[\textbullet] network module for communication between replicas,
  \item[\textbullet] view based recovery (Section \ref{sec:view_ss}),
  \item[\textbullet] disc writers for full stable storage recovery.
\end{tightList}

\noindent The remaining modules were the result of our joint work.
