
\chapter{Introduction}

%Wstęp\footnote{Treść przykładowych rozdziałów została skopiowana
%z ,,zasad'' redakcji prac dyplomowych FCMu~\cite{fcm-red}.} do pracy powinien zawierać następujące elementy:
%\begin{itemize}
%    \item krótkie uzasadnienie podjęcia tematu; 
%    \item cel pracy (patrz niżej); 
%    \item zakres (przedmiotowy, podmiotowy, czasowy) wyjaśniający, w jakim rozmiarze praca będzie realizowana; 
%    \item ewentualne hipotezy, które autor zamierza sprawdzić lub udowodnić; 
%    \item krótką charakterystykę źródeł, zwłaszcza literaturowych; 
%    \item układ pracy (patrz niżej), czyli zwięzłą charakterystykę zawartości poszczególnych rozdziałów; 
%    \item ewentualne uwagi dotyczące realizacji tematu pracy np.~trudności, które pojawiły się w trakcie 
%    realizacji poszczególnych zadań, uwagi dotyczące wykorzystywanego sprzętu, współpraca z firmami zewnętrznymi. 
%\end{itemize}
%
%\noindent
%\textbf{Wstęp do pracy musi się kończyć dwoma następującymi akapitami:}
%\begin{quote}
%Celem pracy jest opracowanie / wykonanie analizy / zaprojektowanie / ...........
%\end{quote}
%oraz:
%\begin{quote}
%Struktura pracy jest następująca. W rozdziale 2 przedstawiono przegląd literatury na temat ........ 
%Rozdział 3 jest poświęcony ....... (kilka zdań). 
%Rozdział 4 zawiera ..... (kilka zdań) ............ itd. 
%Rozdział X stanowi podsumowanie pracy. 
%\end{quote}
%
%W przypadku prac inżynierskich zespołowych lub magisterskich 2-osobowych, po tych dwóch w/w akapitach 
%musi w pracy znaleźć się akapit, w którym będzie opisany udział w pracy poszczególnych członków zespołu. Na przykład:
%
%\begin{quote}
%Jan Kowalski w ramach niniejszej pracy wykonał projekt tego i tego, opracował ......
%Grzegorz Brzęczyszczykiewicz wykonał ......, itd. 
%\end{quote}

%%%%%%%%%%%%%%%%%%%%%%%%%%%%%%%%%%%%%%%%%%%%%%%%%%%%%%%%%%%%%%%%%%%%%%%%%%%%%%%%%%%%%

%    \item krótkie uzasadnienie podjęcia tematu; 
With the increase of usage and importance of Internet services in everyday life comes demand on reliability and constant availability.
Providing these features to the services is a complex matter, often requiring various trade-offs between user satisfaction and service resource demands.

Most common and effective method of providing reliable services is replication. Applying the replication requires only minor difference between non-replicated services and the replicated ones, what makes it usable both for creating new services in the typical way % poprawić zwrot
as well as transforming existing services into replicated ones.


% TODO What do we want to write here?
% Why providing reliability and constant availability is hard? What is causing the problem? 
% fault tolerance

\section{Related work}

Few words about state machine replication, about different approaches (2PC
Two-Phase Commit, 3PC Three-Phase Commit, e3PC Enhanced Three Phase Commit).

Point to first paper about Paxos (The part-time parliament). Then few words about
other papers like Paxos made simple or more engineer papers like Paxos for system
builders or Paxos made live.  

\section{Problem Definition}

Fault tolerant

\section{Our solution}

Implement state machine replication based on Paxos protocol. 

Mention about all modules and performance optimization to make the library ready to use 
in production systems. (Batching, Catch-up, snapshoting)

\section{Organization of the Thesis}
The structure of this bachelor's thesis is following. In chapter 2 the theory about consensus, Paxos algorithm, replication, state machine, failure detector is provided. In chapter 3 our solution is described, Multi Paxos, features like snapshotting, catch-up, failure-detector, recovery algorithms. In chapter 4 the Recovery is described. In chapter 5.

Jan Konczak has done catch-up, recovery from stable storage, ....
Tomasz Zurkowski has done ....
