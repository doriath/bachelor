
\chapter{Introduction}

%Wstęp\footnote{Treść przykładowych rozdziałów została skopiowana
%z ,,zasad'' redakcji prac dyplomowych FCMu~\cite{fcm-red}.} do pracy powinien zawierać następujące elementy:
%\begin{itemize}
%    \item krótkie uzasadnienie podjęcia tematu; 
%    \item cel pracy (patrz niżej); 
%    \item zakres (przedmiotowy, podmiotowy, czasowy) wyjaśniający, w jakim rozmiarze praca będzie realizowana; 
%    \item ewentualne hipotezy, które autor zamierza sprawdzić lub udowodnić; 
%    \item krótką charakterystykę źródeł, zwłaszcza literaturowych; 
%    \item układ pracy (patrz niżej), czyli zwięzłą charakterystykę zawartości poszczególnych rozdziałów; 
%    \item ewentualne uwagi dotyczące realizacji tematu pracy np.~trudności, które pojawiły się w trakcie 
%    realizacji poszczególnych zadań, uwagi dotyczące wykorzystywanego sprzętu, współpraca z firmami zewnętrznymi. 
%\end{itemize}
%
%\noindent
%\textbf{Wstęp do pracy musi się kończyć dwoma następującymi akapitami:}
%\begin{quote}
%Celem pracy jest opracowanie / wykonanie analizy / zaprojektowanie / ...........
%\end{quote}
%oraz:
%\begin{quote}
%Struktura pracy jest następująca. W rozdziale 2 przedstawiono przegląd literatury na temat ........ 
%Rozdział 3 jest poświęcony ....... (kilka zdań). 
%Rozdział 4 zawiera ..... (kilka zdań) ............ itd. 
%Rozdział X stanowi podsumowanie pracy. 
%\end{quote}
%
%W przypadku prac inżynierskich zespołowych lub magisterskich 2-osobowych, po tych dwóch w/w akapitach 
%musi w pracy znaleźć się akapit, w którym będzie opisany udział w pracy poszczególnych członków zespołu. Na przykład:
%
%\begin{quote}
%Jan Kowalski w ramach niniejszej pracy wykonał projekt tego i tego, opracował ......
%Grzegorz Brzęczyszczykiewicz wykonał ......, itd. 
%\end{quote}

%%%%%%%%%%%%%%%%%%%%%%%%%%%%%%%%%%%%%%%%%%%%%%%%%%%%%%%%%%%%%%%%%%%%%%%%%%%%%%%%%%%%%

%    \item krótkie uzasadnienie podjęcia tematu; 
With the increase of usage and importance of Internet services in everyday life comes demand on reliability and constant availability.
Providing these features to the services is a complex matter, often requiring various trade-offs between user satisfaction and service resource demands.

Most common and effective method of providing reliable services is replication. Applying the replication requires only minor changes between non-replicated services and the replicated ones, what makes it usable both for creating new services in the typical way % poprawić zwrot
as well as transforming existing services into replicated ones.

Each replication approach relies on certain group communication mechanism - usually providing strong guarantees. Total-order (atomic) broadcast is a very usefull for interconnecting replicas - it simplifies the design and implementation of replicated protocols, as it alows for dividing the system into two main modules with clearly visible responsibilities.

The services vary, and so the approaches for replication must also vary. A library able to replicate every service would not only be hard to implement, but also could be too complicated for the user.

We chose to implement one of the replication appropach - an approach we trust to have bright future.

We implement active replication of services - that means each copy does the same work. The replicas must work identically, therefore we require the service to behave deterministicly.

Every library for replicating services should adress it's main purpose -- guarantee high availability of service as well as durability of the service work results -- while having minimum requirements for the enviornment.

Our software design goal is to work well in real networks, survive temporary network crashes and be able to recover from replica crashes as well.

{
\bfseries
The goal of this bachelor's thesis is to create Java library implementing state machine replication primitive. The library should support the crash-recovery model of failure and should tolerate message loss and communication delays.
}

\begin{TODO}

\section{Related work}
Literature

\begin{itemize}
  \item The part-time pariliament
  \item Paxos made simple
  \item Paxos made live
  \item Paxos for system builders
\end{itemize}

\end{TODO}

\begin{TODO}
The organization of this thesis is following. In chapter 2 the background for distributed computing is provided. We describe what is consensus problem, etc. In chapter 3 we describe the architecture and modules of JPaxos.
\end{TODO}

Jan Konczak for the purpose of this thesis has prepared design and implementation of:
\begin{itemize}
  \item catch-up module (Section \ref{sec:catch_up}),
  \item snapshotting module (Section \ref{sec:snapshotting}),
  \item service proxy (Section \ref{sec:service_proxy}),
  \item benchmark project.
\end{itemize}

Tomasz Zurkowski has prepared design and implemenation of:
\begin{itemize}
  \item unit tests for JPaxos project,
  \item client manager module, 
  \item network module for communication between replicas (Section \ref{sec:network}),
  \item view based recovery (Section \ref{sec:view_ss}),
  \item disc writers for full stable storage recovery.
\end{itemize}

The remaining modules were result of our common work, and later division of work is just infeasible.
