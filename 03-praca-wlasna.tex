
\chapter{JPaxos}

\section{System model}

We assume that we replicate N servers - called processes in later discussion.
Servers communicate with each other using asynchronous communication with no
guarantees - messages can be lost and duplicated. Each server is running the
same state machine which is executing all requests from the clients.

\section{Paxos Protocol}

- leader election phase (prepare, prepare ok)
- propose / accept phase

\section{Leader election}

Every process keeps track of most recent view number received from any of other
process. View number is inside every message sent between processes. The view
number is used to determine the current leader. The process is a leader if: 
$$ view \bmod N \equiv process\_id $$
where:

\begin{description}
  \item[view] - current view number
  \item[N] - number of replicated servers (processes)
  \item[process\_id] - the unique process id in range 
\end{description}

\section{MultiPaxos}

Multiple instances.

\section{Architecture}

\section{Batching}

\section{Catch-up mechanism}

\section{Recovery algorithms}

\subsection{Crash Stop}

\subsection{Crash-recovery with stable storage}

\subsection{View based recovery}

\subsection{Epoch based recovery}

%Rozdziały dokumentujące pracę własną studenta: opisujące ideę, sposób lub metodę 
%rozwiązania postawionego problemu oraz rozdziały opisujące techniczną stronę rozwiązania 
%--- dokumentacja techniczna, przeprowadzone testy, badania i uzyskane wyniki. 
%
%Praca musi zawierać elementy pracy własnej autora adekwatne do jego wiedzy praktycznej uzyskanej w
%okresie studiów. Za pracę własną autora można uznać np.: stworzenie aplikacji informatycznej lub jej
%fragmentu, zaproponowanie algorytmu rozwiązania problemu szczegółowego, przedstawienie projektu 
%np.~systemu informatycznego lub sieci komputerowej, analizę i ocenę nowych technologii lub rozwiązań
%informatycznych wykorzystywanych w przedsiębiorstwach, itp. 
%
%Autor powinien zadbać o właściwą dokumentację pracy własnej obejmującą specyfikację założeń i 
%sposób realizacji poszczególnych zadań
%wraz z ich oceną i opisem napotkanych problemów. W przypadku prac o charakterze 
%projektowo-implementacyjnym, ta część pracy jest zastępowana dokumentacją techniczną i użytkową systemu. 
%
%W pracy \textbf{nie należy zamieszczać całego kodu źródłowego} opracowanych programów. Kod źródłowy napisanych
%programów, wszelkie oprogramowanie wytworzone i wykorzystane w pracy, wyniki przeprowadzonych
%eksperymentów powinny być umieszczone na płycie CD, stanowiącej dodatek do pracy.
%
%\section*{Styl tekstu}
%
%Należy\footnote{Uwagi o stylu pochodzą częściowo ze stron Macieja Drozdowskiego~\cite{mdro}.} 
%stosować formę bezosobową, tj.~\emph{w pracy rozważono ......, 
%w ramach pracy zaprojektowano ....}, a nie: \emph{w pracy rozważyłem, w ramach pracy zaprojektowałem}. 
%Odwołania do wcześniejszych fragmentów tekstu powinny mieć następującą postać: ,,Jak wspomniano wcześniej, ....'', 
%,,Jak wykazano powyżej ....''. Należy unikać długich zdań. 
%
%,,Ilość'' i ,,liczba''. Proszę zauważyć, liczba dotyczy rzeczy policzalnych, np.~liczba osób, liczba zadań, procesorów. 
%Ilość dotyczy rzeczy niepoliczalnych, np.~ilość wody, energii. Należy starać się wyrażać precyzyjnie, tj.~zgodnie 
%z naturą liczonych obiektów.\footnote{(DW) Według wytycznych Rady Języka Polskiego obie formy są dopuszczalne
%zarówno do obiektów policzalnych, jak i niepoliczalnych. W tekstach technicznych warto być jednak precyzyjnym.}
%
%Niedopuszczalne są zwroty używane w języku potocznym. W pracy należy używać terminologii informatycznej, która ma 
%sprecyzowaną treść i znaczenie. Nie należy używać ,,gazetowych'' określeń typu: 
%silnik bazy danych, silnik programu, maszyna skryptowa, elektroniczny mechanizm, mapowanie, string, gdyż nie wiadomo 
%co one właściwie oznaczają. 
%
%Niedopuszczalne jest pisanie pracy metodą \emph{cut\&paste}, bo jest to plagiat i dowód intelektualnej indolencji autora.
%Dane zagadnienie należy opisać własnymi słowami. Zawsze trzeba powołać się na zewnętrzne źródła. 

