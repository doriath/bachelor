\documentclass[hyperref={pdfpagelabels=true},11pt,compress,trans]{beamer}


\usepackage[OT4]{fontenc}
\usepackage[utf8x]{inputenc}

\usepackage{amsfonts}
\usepackage{amsmath}
\usepackage{algorithm}
\usepackage{array}
\usepackage{beamerthemeboxes}
\usepackage{fancyvrb}
%\usepackage{beamerthemeshadow}
\usepackage{subfigure}

\usepackage{listings}
\lstset{
  language=java,
  basicstyle=\footnotesize,
  numberstyle=\footnotesize,
  tabsize=2,
  breaklines=true
  }

\setbeamertemplate{navigation symbols}{}
\setbeamertemplate{headline}{}

\usetheme{Madrid}

%%% Available themes:
% AnnArbor Antibes Bergen Berkeley Berlin Boadilla boxes
% CambridgeUS Copenhagen Darmstadt default Dresden Frankfurt
% Goettingen Hannover Ilmenau JuanLesPins Luebeck Madrid
% Malmoe Marburg Montpellier PaloAlto Pittsburgh Rochester
% Singapore Szeged Warsaw
%%%

\usecolortheme[rgb={0.2,0.2,0.2 }]{structure}
\definecolor{myGray}{cmyk}{0,0,0,0.4}
\setbeamercolor*{palette quaternary}{fg=gray,bg=myGray}

%%% Some examle, copied from my other presentation colour settings
%\usetheme{Warsaw}
%\usecolortheme[rgb={0.25,0.8,0.25}]{structure}
%\definecolor{ciemnozielony}{RGB}{20,60,20}
%\setbeamercolor*{palette quaternary}{fg=white,bg=ciemnozielony}
%%% Other colour settings
% \usecolortheme[rgb={1,0.2,0}]{structure}
% \definecolor{cyan}{cmyk}{100,0,0,0}
% \definecolor{bordowy}{cmyk}{0,100,0,0.25}
% \setbeamercolor{frametitle}{fg=bordowy,bg=cyan}
% \definecolor{sjena_spalona}{RGB}{120,80,40}
% \definecolor{zzielenialy}{RGB}{80,255,120}
% \setbeamercolor*{normal text}{fg=zzielenialy,bg=sjena_spalona}
% \setbeamercolor*{palette quaternary}{bg=orange}
%%%

\newenvironment{TODO}
  {\begin{center}\par\noindent
    {\huge \vspace{-2em}\par\noindent  \rule{0.55\textwidth}{0.05em}  \\ \nointerlineskip
    \[ \mathbf{T} \, _\mathcal{O} \:\: \mathfrak{D} \, _\mathbb{O} \] }
    \begin{minipage}{0.5\textwidth}
  }{\end{minipage} {\huge \rule{0.55\textwidth}{0.05em}} \par\noindent \end{center} \par}

\newenvironment{warning}{{\rmfamily\textsc{Warning:}}\it} {\normalfont}

\newcommand{\mySection}[1]
{
  \section{#1}
  \begin{frame}{}
    \rmfamily\Large
    \begin{center}
        \par\noindent\ignorespaces\rule{0.9\textwidth}{0.05em} \\
        \textsc{#1} \\ \vspace{-\parskip}
        \rule{0.9\textwidth}{0.05em}
    \end{center}
  \end{frame}
}

\setbeamertemplate{block begin}
  {\par\vskip\medskipamount{\noindent\!\!\!\underline{\rmfamily\textsc{\insertblocktitle}}}\par\noindent\ignorespaces}
\setbeamertemplate{block end}
  {\vskip\medskipamount}

\setbeamertemplate{frametitle}
{
  \nointerlineskip
  \begin{beamercolorbox}[sep=0.2cm,wd=\pdfpagewidth]{frametitle}
    \usebeamerfont{frametitle}
    \vbox{}\vskip-1ex
    \strut\rmfamily\textsc{\hspace{0.3cm}\insertframetitle}\strut\par
      \usebeamerfont{framesubtitle}\usebeamercolor[fg]{framesubtitle}\rmfamily\textsc{\hspace{0.3cm}\insertframesubtitle}\strut\par
    \vskip-1ex
    \vskip-.3cm
  \end{beamercolorbox}
}

\useinnertheme{default}
\setbeamertemplate{enumerate items}[default]
\setbeamertemplate{itemize items}[default]
\setbeamersize{text margin left=2em,text margin right=2em}

\setlength{\parskip}{0.5em}

\usefonttheme{professionalfonts}

\title{JPaxos}
\date{\today}
\author{Jan Ko\'nczak, Tomasz \.Zurkowski}

\begin{document}


{ %%% Titleframe
  \setbeamertemplate{footline}{}
  \setbeamertemplate{frametitle}{}

  \logo
  {
    \includegraphics[keepaspectratio,height=0.2\textheight]{images/put_logo.pdf} \hspace*{3.5em}
  }

  \begin{frame}
    \rmfamily
    \begin{center}
      \Huge
        \par\noindent\ignorespaces\rule{0.8\textwidth}{0.05em} \\
        \textsc{JPaxos}

        \vspace*{-\parsep}
        \begin{minipage}{0.5\textwidth}
          \normalsize\centering
          \textsc{State~machine~replication\\based~on~the~Paxos~protocol}
        \end{minipage}

        \rule{0.8\textwidth}{0.05em}
      \tiny

      \vspace{1em}

      \begin{columns}
       \column{0.4\textwidth}
       \column{0.4\textwidth}
        \begin{tabbing}
          \textsc{Authors} \qquad \= \\
          \>Jan \' Kończak \\
          \>Tomasz \' Żurkowski \\
          \textsc{Thesis advisor} \\
          \> PhD Paweł \' T. Wojciechowski
        \end{tabbing}
      \end{columns}
    \end{center}
  \end{frame}
}


  \logo
  {
    \includegraphics[keepaspectratio,height=0.12\textheight]{images/put_logo.pdf} \hspace*{1em}
  }

\begin{frame}{Tasks}
 \begin{block}{IT-SOA (March 2009 - June 2010)}
  Full Paxos implementation supporting crash-stop model
 \end{block}

 \begin{block}{Tasks for thesis}
  Optimizing JPaxos
  
  Preparing code to meet crash-recovery requirements
  
  Adding various crash-recovery models: \vspace{-\parsep}
  \begin{itemize} 
   \item Full Stable Storage
   \item Epoch-based
   \item View-based
  \end{itemize}

  Testing
 \end{block}
\end{frame}

\begin{frame}{System assumptions}
  \begin{block}{Model}
    \vspace{-1em}\vspace{-\parskip}\vspace{-\lineskip}
    \begin{itemize}
      \item Non-byzantine behaviour
      \item Static group membership
      \item Crash-recovery
      \item Catastrophic failures (may be) tolerated
     \end{itemize}
  \end{block}

  \begin{block}{Network}
    \vspace{-1em}\vspace{-\parskip}\vspace{-\lineskip}
    \begin{itemize}
      \item Messages may get lost
      \item Large delays acceptable for bounded time
    \end{itemize}
  \end{block}
\end{frame}

\begin{frame}{Guarantees}
  \begin{block}{State machine replication}
    \vspace{-1em}\vspace{-\parskip}\vspace{-\lineskip}
    \begin{itemize}
      \item If majority of replicas will be correct, system is able to decide
      \item All client requests will be executed on each state machine exactly once (even if it will be sent multiple times)
      \item All state machines will execute requests in the same order
      \item After crashing, replica will recover from other replicas to actual state
    \end{itemize}
  \end{block}
\end{frame}

\begin{frame}{Features implemented before summer 2010}
 \begin{block}{}
    \vspace{-1em}\vspace{-\parskip}\vspace{-\lineskip}
    \begin{itemize}
      \item Failure detector
      \item Networking
      \item Client management
      \item Catching up
      \item Snapshotting
      \item Batching
      \item Concurrent instances
      \item Stable storage writes
    \end{itemize}
  \end{block}
\end{frame}

\begin{frame}{Transparency for the user}
 \begin{block}{Service vs JPaxos}
  For the target service a request is the smallest data portion
  
  For the replica an instance is the smallest data portion
  
  The instance may have multiple requests
 \end{block}

 \begin{block}{ServiceProxy}
  Translates from requests to instances and vice versa
  
  Cares for proper snapshot creation and recovery
 \end{block}
\end{frame}

\begin{frame}{Full Stable Storage}
 \begin{block}{Assumption}
  Use stable storage (hard drive) to log all possible stray messages
  
  Read the contents of Stable Storage on recovery
 \end{block}

 \begin{block}{Features}
  \vspace{-1em}\vspace{-\parskip}\vspace{-\lineskip}
  \begin{itemize}
   \item[+]  Replica forgets nothing
   \item[+]  No additional communication steps
   \item[+]  Catastrophic failures are acceptable
   \item[--] Catastrophic slowdown
   \item[--] Needs synchronous disk writes
  \end{itemize}
 \end{block}
\end{frame}

\begin{frame}{Epoch-based algorithm}
 \begin{block}{Assumption}
  Differentiate messages by consecutive run number (epoch)

{\footnotesize i.e. First run has number 1, after first crash and recovery replica has run number 2}

  Replicas will ignore stray messages, as they will recognize them

  By recovery the JPaxos state is taken from other replicas
 \end{block}
 \begin{block}{Features}
  \vspace{-1em}\vspace{-\parskip}\vspace{-\lineskip}
  \begin{itemize}
   \item[+]  Fast
   \item[+]  No additional communication steps
   \item     Majority of replicas must be anytime up
   \item     Stable Storage is used once per run
  \end{itemize}
 \end{block}
\end{frame}

\begin{frame}{View-based algorithm}
 \begin{block}{Assumption}
  Every message is labelled with view - let's force the view to be bigger than any stray message may have

  Replicas will ignore stray messages, as they will consider them stale

  By recovery the JPaxos state is taken from other replicas
 \end{block}
 \begin{block}{Features}
  \vspace{-1em}\vspace{-\parskip}\vspace{-\lineskip}
  \begin{itemize}
   \item[+]  Fast
   \item[+]  No additional communication steps
   \item     Majority of replicas must be anytime up
   \item     Stable Storage is used once per view change
  \end{itemize}
 \end{block}
\end{frame}

{
 \setbeamertemplate{footline}{}
 \setbeamertemplate{frametitle}{}
 \logo
{
  \includegraphics[keepaspectratio,height=0.2\textheight]{images/put_logo.pdf} \hspace*{3.5em}
}

\begin{frame}{}
  \rmfamily\Large
  \begin{center}
      \par\noindent\ignorespaces\rule{0.9\textwidth}{0.05em} \\
      \textsc{Thank You for Your attention} \\ \vspace{-\parskip}
      \rule{0.9\textwidth}{0.05em}
  \end{center}
\end{frame}

}
\end{document}


