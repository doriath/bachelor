\chapter{Pseudocode of Paxos algorithm}

In this chapter, we present the pseudocode of Paxos algorithm written by us to better understand the algorithm, before writing implementation in Java. The pseudocode was designed and written within IT-SOA project, before the internship at EPFL.

\begin{lstlisting}[frame=lines,caption=Pseudocode of Paxos algorithm]

=================================================
                Classes, structs and data types planned to use
=================================================

/// As process ID we're planning to use a single unsigned int value
typedef uint proc_id;

/// The same counts for decision number
/// The vote_t has also a special value called NULL, set as zero (the votings
/// will begin at least with 1)
typedef uint vote_t;

/// Special "NoVote" value, meaning either no voting, or no earlier promise.
#DEFINE NoVote 0

/// Special value no operation - if all processes having a value are down,
/// we need to ensure no-op will be chosen
#DEFINE NOOP

/// Type for values (which also mean state machine operations)
struct value_t
{
	uint size;
	char * data;
};

/**
 * generic class map assumptions: it's a association table, that means a table
 * indexed by some key. we have [] operator, has value and remove function.
 * At the beginning each map is empty, writing to non-existing element creates
 * it. Some func gives us count.  Most languages have something like that;
 * in C++ it's std::map, in C# it's Dictionary, apparently all descendants
 * of the Map Interface in Java etc.
 **/
template<typename T1, typename T2> class map;


template<typename T1> class set;


/// My process ID - given by joining the group.
proc_id myProcessID = 0;

/**
 * \struct Data
 * Data of the message should contain some value for state machine 
 * and also the client, which send this message. The client is needed
 * because state machine has to know where to send the response to the 
 * message.
 **/
struct Data
{
	char * value;
	Client client;
};

/**
 * \struct Message
 * Proposed representation of the paxos message.
 **/
struct Message
{
	proc_id sender; ///< ID of sending process
	enum messageType ///< Type of the massage
	{
		ALIVE   = 0x00, // For Failure Detector and once used as meaningless message
		PREPARE = 0x01,
		PROMISE = 0x02,
		VOTE    = 0x04,
		VOTED   = 0x08,
		SUCCEED = 0x10,
		NACK    = 0x20, // A not-acknowledge message to make the algorithm faster
		OTHER   = 0xFF, // Yet undefined ;-)
	} type;
	
	uint instanceId; ///< The number of paxos instance
	uint size; ///< Size of the message
	Data data; ///< All the other data 
};

/**
 * \enum Status
 * Indicates the state in which every SinglePaxosEntity currently is.
 **/
enum Status
{
	IDLE,   // the default status of each process
	        // indicates that process try to start new ballot 
	        // (sending prepare messages and waiting for promise responses)

	TRYING, // indicates that process started new ballot
	        // (sending vote messages and waiting for voted responses)

	POLLING 
};

/**
 * \class FailureDetector
 * This class will have current list of all members, with time of 
 * last received message from them. 
 **/
class FailureDetector
{
	public:
		/// By initiating, we do exactly nothing
		FailureDetector(){}
		
		/// Checks, if a process sent anything for the last {timeout} seconds
		isAlive(proc_id id);
		/// Returns lowest process id which is alive
		getLowestAliveId();
		
		/// Class GroupMembers tells the failure detector that the group has
		/// changed
		addMember(proc_id id);
		removedMember(proc_id id);
		
		/// This list holds time of last message from processes, for example as
		/// unix timestamps (seconds elapsed from 01.01.1970)
		map<proc_id, time_t> lastMessageReceived;
		
		/// Time after which a process is thought to be dead
		static const uint timeout;
};


/**
 * \class GroupMembers
 * This class will have current list of all members, that means it will inform
 * whether a process is at that moment taking part in consensus.
 *
 * The list can be modified by the state machine itself, according to the pro-
 * per rules.
 **/
class GroupMembers
{
	public:
		/// Initializes new instance of GroupMembers with specified failure detector.
		/// The specified failure detector will be notified about every change of group 
		/// members.
		///
		/// By initialization we should request group membership; currently we're reading
		/// only a config file.
		GroupMembers(FailureDetector & fd);
	
		list<proc_id> currentMembers;
		isMember(proc_id id);

		addMember(proc_id id); ///< This function also inform failure detector!
		removeMember(proc_id id); ///< This function also inform failure detector!

		getMemberCount();
		getMajority()
	protected:
		FailureDetector & failureDetector;
};

/**
 * \class MessageReceiver
 * Working as a wrapper for incoming messages this class will record time for 
 * every message, check it's type, sender and forward it to Paxos protocol.
 **/
class MessageReceiver
{
	public:
		MessageReceiver(FailureDetector & fd, GroupMembers & gm);
		
		/// Broadcasts message to all members of current group.
		/// Note: This doesn't has to be a reliable broadcast. We cannot
		/// assume that all members will receive this message.
		broadcastMessage(Message msg);
		
		/// Sends message to member with specified id
		sendMessage(proc_id receiver, Message msg);
		
		/// This function will add new listener for incoming messages. The 
		/// listener is function with one argument (message) and returning 
		/// nothing (void). All listeners are executed when new message was
		/// received.
		/// example:
		///   void receiveMessage(Message msg);
		///   messageReceiver.addListener(receiveMessage);
		addListener(function func);
		removeListener(function func);
	protected:
		FailureDetector & failureDetector;
		GroupMembers & groupMembers;
	
		/// This function is designed to receive every message, and after that
		/// forward it if necessary to the Paxos algorithm
		receiveMessage(Message m);
		
		/// The list of all listeners waiting for incoming messages
		list<function> listeners;
};

/**
 * \class MultiPaxos
 * This class is the main class managing whole consensus process.
 * It creates single paxos instances, forwards them appropriate messages etc.
 * If the process is the leader, it must also conduct whole process.
 **/
class MultiPaxos
{
	public:
		MultiPaxos(FailureDetector & fd,
				   MessageReceiver & mr,
				   GroupMembers & gm,
				   StableStorage & ss,
				   StateMachine & sm);
	
		receiveMessage(Message m);
		/// After switching on multiPaxos is informed how many entries had the
		/// file, that is how many entries were read after crash
		fromFile(int how_many_read);
		
		/// Stops current algorithm clearly.
		stop();
	protected:
		FailureDetector & failureDetector;
		MessageReceiver & messageReceiver;
		GroupMembers & groupMembers;
		StableStorage & storage;
		StateMachine & stateMachine;
		
		// The time after which check method will be called in all active paxos instances.
		// static const uint timeout; should be taken from FaliureDetector::timeout
		
		
		/// Determines whether this algorithm is running. This is used to 
		/// stop algorithm clearly.
		bool isRunning;
		
		/// The main loop started in some other thread, used to monitor current
		/// state of single instances. It is important due to crashes of the leader,
		/// and when we have to repeat some paxos algorithm phase. This also fill
		/// not finished instances with NOOP value.
		thread mainLoop
		{
			uint checksPerTimeout = 3;
			run();
		};
		
		/// Gets the process id of current leader. This is lowest id from alive processes
		proc_id getLeader();
		
		/// Determines whether current process is the leader
		bool amILeader();
		
		/// Adds new Paxos instances if necessary; otherwise does nothing
		addMissingPaxos(Message msg)
		
		/// runningPaxos has the list of single Paxos instances started until
		/// this moment. As first message emerges, new instance is created and
		/// added here
		list<SinglePaxosEntity> runningPaxos;
		
		/// The list of instanceId of paxos instances which are not finished yet.
		set<uint> activePaxos;
};

/**
 * \class SinglePaxosEntity
 * Is a single, typical Paxos algorithm. It gets from MultiPaxos only messages
 * for the instance, and takes a single decision.
 **/
class SinglePaxosEntity
{
	public:
		SinglePaxosEntity(StableStorage::SingleEntry my_entry, MessageReceiver & mr, Message msg, uint instanceId)
		receiveMessage(Message msg);
	protected:
		SingleEntry & entry;
		MessageReceiver & messageReceiver;
		
		/// The id of current paxos instance. This is needed to set proper
		/// instance id when sending message to other members (e.g. prepare message)
		uint instanceId;
	
		// The amount of time used to determine whether some phase should be repeated.
		// const uint timeout; should be taken from FaliureDetector::timeout
		
		/// The last time when some message related with specified phase
		/// was received. This times are used to determine whether some 
		/// phase should be repeated
		uint lastTryingTime;
		uint lastPoolingTime;
		
		uint lastMessageTime;
	
		/// This method is called every amount of time, to check crashes.
		/// This is responsible, to start prepare phase if old leader crashes,
		/// and also to repeat some phase if not enough responses is get in 
		/// specified amount of time.
		check(bool amILeader);
	
		/// the methods executed when appropriate message will be received
		prepare(Message & msg);
		promise(Message & msg);
		vote(Message & msg);
		voted(Message & msg);
		succeed(Message & msg);
		
		/// Routine for not acknowledge handling
		nackReceived(Message & msg);
		
		/// Starts the prepare phase, and set current state to TRYING. This 
		/// method sends the prepare message to all processes.
		sendPrepare();
		
		/// Starts the voting phase, and set current state to POOLING. This
		/// method setds the vote message to all processes.
		sendVote();
		
		/// the variables which can be in RAM, and that not a problem if it will
		/// be lost
		
		/// The current status of this paxos instance
		Status status;
		
		/// the set promisers and voters is used by leader only
		
		/// indicates processes from which promise messages was received for lastTriedBallot
		set<proc_id> promisers; 
		/// indicates processes from which voted messages was received for lastTriedBallot
		set<proc_id> voters;
		
		/// The values sent by proccesses in responce to prepare message.
		set<pair<vote_t, value>> valuesSentByPromisers;
		
		/// The message from the client which will be forced when no other process
		/// has some value in first phase.
		Message clientMessage;
};

/**
 * \class StateMachine
 * Represents the state machine. When we run the same state machine on different 
 * processes and execute the same set of methods, in the same order, then on each 
 * process the state machine will be in the same state. This class also manage to
 * check which values can be executed, and when we have to wait for some previous 
 * values.
 */
class StateMachine
{
	public:
		StateMachine(MessageReceiver &  mr);
		
		addValue(Data & data, int instanceId);
		
	protected:
		/// Message receiver used to send response to the client.
		MessageReceiver & messageReceiver;
		
		/// Indicates how many values has been executed on this machine.
		/// This will be always values with numbers from 1 to lastCoherentPart.
		/// Each executed values has to be succeeded in paxos.
		int lastCoherentPart = 0;
		
		/// The data which are currently succeeded on paxos, and are waiting 
		/// to be executed.
		map<int, Data> commands;
		
		/// Executes some data on this state machine.
		execute(Data data);
}

/**
 * \class StableStorage
 * We implement the stable storage as two disk files.
 * One of them has triples of constant-length fields:
 *  - if the consensus succeeded
 *  - last promise value
 *  - last ballot number
 *  - last value number
 *  - offset to the last value
 *  - size of the last value
 * As for the offset: in order to keep files simple, we need to make some
 * easy to implement structure. Thanks to constant length of one entry, we
 * can problemlessly use it without shifting larger part of the file.
 *
 * As for the voting values: initially, they will be empty - we will use any
 * offset for that, and indicate no value with 0 size. As we receive first
 * message with some value, we'll fill offset and size after writing value to
 * the file. Even if we will receive other value (what is a very rare situation
 * because only one process proposes it's value, so we should either accept
 * it, or take an no-op) we can simply by 'loosing' some place skip moving
 * large part of the file. It's acceptable, because we can ensure really small
 * amount of conflicts (in fact, we must do in order to preserve progress).
 **/
class StableStorage
{
	public:
		/// Structure needed by each Paxos instance
		class SingleEntry
		{
			public:
				SingleEntry(uint num, int & offset);
			protected:
				int & lastOffset; /// Reference to value needed for valuesFile
				const number; /// Instance number
				bool succeeded;
				vote_t lastTriedBallot; // The last number of ballot which this process began
				vote_t lastVotedBallot; // The last number of ballot in which this process voted
				vote_t lastPromise; // The last number of ballot which this process promise not to accept lesser one
				int offsetToValue;
				unsigned sizeOfValue;
			public:
				/// The getters simply read value from RAM
				getSucceded() {return succeeded;}
				getLastTriedBallot() {return lastTriedBallot;}
				getLastVotedBallot() {return lastVotedBallot;}
				getLastPromise() {return lastPromise;}
				getValue();
				{return readFile(valuesFile,offsetToValue,size);}

				/// The setters need to write first our function on hard drive
				/// and after then to update RAM
				setSucceded(bool);
				setLastTriedBallot(vote_t);
				setLastVotedBallot(vote_t);
				setLastPromise(vote_t);
				setValue(value_t);
		};

	protected:
		/// Vector (a table) of data stored in RAM for fast read operation
		vector<SingleEntry> paxosData;

		/// Filenames; these need to be constant (or known) in order to allow
		/// crash-recovery. Some configuration file can be later issued
		///
		/// As for pseudocode, these names also indicate open files, that means
		/// writeToFile(mainFile, offset, value) is a allowed write operation.
		/// In real code we'll need at least two variables for this.
		///
		/// stateFile has only number of entries yet initiated;
		/// mainFile has four higher described numbers;
		/// valuesFile has the values.
		///
		/// stateFile is in fact unnecessary, if we can read size of mainFile,
		/// as the record length is constant.
		const string stateFile;
		const string mainFile;
		const string valuesFile;

		int lastOffset;

	public:
		/// In order to simplify the (pseudo)code, we allow to use this class
		/// as typical table of singleEntry objects
		SingleEntry operator[](uint num)
				{ if(paxosData.count()<num) return paxosData[num]; return 0; }

		/// This functions will add objects for new consensus instances
		addNewEntry();
		addNewEntries(int n) { for(n times) addNewEntry();}
		addNewEntriesUpTo(int n) { for( (n-paxosData.count()) times) addNewEntry();}

		/// Returns the number of data == the number of instances
		count() { return paxosData.count(); }

		/// The initializing function must read from file previous data. This
		/// lets crash-recovery; if process crashed, it must read previous
		/// state (if any).
		StableStorage();

};

/**
 * \class Heartbeat
 * This class simply sends every {timeout}/{countPerTimeout} a message treated
 * as 'alive indicator' for all other processes.
 **/
class Heartbeat
{
	public:
		/// This class needs information what group should be informed, and
		/// which class will handle message sending.
		Heartbeat(MessageReceiver & mr, groupMembers & gm);
		
		uint countPerTimeout = 3;
	
	protected:
		list<proc_id> & currentMembers;
		
		MessageReceiver & messageReceiver;
		
		Message mes;
		
		bool isRunning = true;
		
		/// Additional thread OR some scheduling trick is needed in order to
		/// assure the message is sent
		thread heartBeater
		{
			/// Func executed by the thread, here: sending the messagages in
			/// proper moments
			run();
		}
};



==================================================
                                Message handling
==================================================

UPON(receiving message m) execute receiveMessage(Message m) at MessageReceiver

==================================================
                            Pseudo-code of functions
==================================================


-------------------------------------------------
                       Message receiver && failure detector
-------------------------------------------------

MessageReceiver::MessageReceiver(FailureDetector & fd, GroupMembers & gm)
{
	failureDetector = fd;
	groupMembers = gm;
	
	// ### in real implementation we have to implement listening for 
	// ### new incoming messages, and the call receiveMessage
}

MessageReceiver::broadcastMessage(Message msg)
{
	// sending message to all members of current group.
	// 1st way: we iterate through all members of group and send
	// message to them (for TCP protocole)
	// 2nd way: we multicast message (for UDP-multicast)
	
	for(id in gm.currentMembers)
		sendMessage(id, msg);
	OR
	multicast(gm.currentMembers, msg);
}

MessageReceiver::sendMessage(proc_id receiver, Message msg)
{
	// language specific code for sending message
	// it can use some internet protocol like TCP or UDP
	sendMessage(receiver, msg);
}

MessageReceiver::addListener(function func)
{
	listeners.add(func);
}

MessageReveiver::removeListener(function func)
{
	listeners.remove(func);
}
			   
MessageReceiver::receiveMessage(Message m)
{
	// ### check procedure if the message is not from some stupid port scanner
	// ### should go here. As for now, a totally non-byzantine model is assumed


	// If the sender is unknown, we need to check if it isn't in the group
	// (but, in fact, this is unnecessary, the groupMembers should inform
	// us that a new member joined the group
	if ( ! lastMessageReceived.hasKey( m.sender ) )
	{
		if ( ! groupMembers.isMember( m.sender ) )
		{
			drop message;
			return;
		}
	}

	// The last reception time is updated to current system time
	lastMessageReceived[m.sender] = time(0);

	// The type of message is checked, and proper action taken up
	switch ( m.type )
	{
		// If it's heart beat, we should do nothing else
		case ALIVE:
			return;

		// If this is one of consensus messages, we forward it
		case PREPARE | PROMISE | VOTE | VOTED | SUCCEED:
			// notify all listeners about new message
			for(listener in listeners)
				listener(m);
			break;
			
		default:
			// ### If we receive a unknown message, that means something has
			// ### gone wrong - some reaction should be implemented here
	}
}

-------------------------------------------------
							    Failure detector
-------------------------------------------------

FaliureDetector::isAlive(proc_id id)
{
	// the process is suspected to be down when it didn't sent any message
	// in last {timeout} seconds
	return lastMessageReceived[id] - time(0) < timeout;
}

FaliureDetector::addMember(proc_id id)
{
	// We indicate that the process is in group by creating it's slot in map
	// 1st way: we set last time for current time minus timeout - it means
	// we would consider the process for dead. Bad if timeout is't constant
	// (we assume it is)
	// 2nd way: if we use unix time, setting timestamp to 0 would mean setting
	// it to the oldest date we can.
	lastMessageReceived[id] = time(0) - timeout;
	OR
	lastMessageReceived[id] = 0;
}

FaliureDetector::removeMember(proc_id id)
{
	// We simply delete the record. It's even unnecessary to stop tracking
	// this processes (it doesn't take additional time), but many not deleted
	// process id's would mean longer search and bigger memory usage.
	lastMessageReceived.remove(id);
}

FailureDetector::getLowestAliveId()
{
	proc_id lowest = null;
	for(pair<proc_id, time_t> procTime in lastMessageReceived)
	{
		if(lowest == null || lowest > procTime.first)
			lowest = procTime.first;
	}
	return lowest;
}

-------------------------------------------------
						          Multi Paxos
-------------------------------------------------

MultiPaxos::MultiPaxos(FailureDetector & fd,
					   MessageReceiver & mr,
				       GroupMembers & gm,
				       StableStorage & ss,
					   StateMachine & sm);
{
	failureDetector = fd;
	messageReceiver = mr;
	groupMembers = gm;
	storage = ss;
	stateMachine = sm;
	
	// every time there is new message, receiveMessage is executed
	messageReceiver.addListener(receiveMessage);
	
	{ // Here begins recovery process
		Message msg; 
		msg.type = ALIVE;
		msg.instanceId = ss.count(); // We get number of paxoses read from file
		
		addMissingPaxos(msg); // We create the paxoses
		// The main loop will in short time start it's work. It'll automaticly 
		// get all newer values and fill the missing ones.
	}
	
	
	for(int i=1; i<=activePaxos.count(), ++i)
	{
		if(ss[i].getSucceded())
		{
			activePaxos.remove(i);
			stateMachine.addValue(ss[i].getValue());
		}
	}
	
	// start main loop
	isRunning = true;
	
	// Running the main loop thread
	mainLoop.run();
}			   

MultiPaxos::receiveMessage(Message m)
{
	if( ! isRunning) return; // If the algorithm is not running, no message may be forewarded
	
	// is this a message from a client
	if(m is message from client)
	{
		if(m is duplicate message)return;
		
		// if i am not leader I have to send it to current leader
		if(amILeader() == false)
		{
			send(getLeader(), m);
			return;
		}
		m.instanceId = (biggest instanceId from runningPaxos) + 1;
		
		addMissingPaxos(m);
		
		return;
	}
	
	// is this message used by paxos algorithm
	if(m is a paxos message)
	{
		addMissingPaxos(m);
		runningPaxos[m.instanceId].receiveMessage(m);
		
		// if this is succeed message, then we can add this data to state machine
		if(m.messageType == SUCCEED)
		{
			stateMachine.addValue(m.data, m.instanceId);
			activePaxos.remove(m.instanceId);
		}
	}
}

MultiPaxos::addMissingPaxos(Message msg)
{
	// get biggest instance id among running paxos (runningPaxos is a set of numbers)
	uint lastInstanceId = runningPaxos.max();
	
	// If the instance(s) already exist, return
	if(msg.instanceId <= lastInstanceId) return;
	
	
	ss.addNewEntriesUpTo(m.instanceId);
	
	// create all missing paxos
	for(int i = lastInstanceId + 1; i < m.instanceId; i++)
	{
		// message used to initializes paxos instance properly
		Message msg; 
		msg.type = ALIVE;
		msg.instanceId = i;
		
		runningPaxos[i] = new SinglePaxosEntity(ss[i], messageReceiver, msg);
		activePaxos.add(i);
	}
	
	activePaxos.add(m.instanceId);
	runningPaxos[m.instanceId] = new SinglePaxosEntity(ss[m.instanceId], messageReceiver, m);
}

// The starting routine for mainLoop thread
MultiPaxos::mainLoop::run()
{
	while(isRunning)
	{
		bool leader = amILeader();
		foreach(proc_id id, activePaxos)
		{
			runningPaxos[id].check(leader);
		}
		
		// checksPerTimeout is a thread value
		this.suspend( (double) FaliureDetector::timeout / checksPerTimeout);
	}
}

MultiPaxos::amILeader()
{
	return ( getLeader() == myProcessID );
}

MultiPaxos::getLeader()
{
	return failureDetector.getLowestAliveId();
}
		
MultiPaxos::stop()
{
	isRunning = false;
	
	// TODO: Leaving group should go here.
	//       Flushing disk files may also go here
	//       For further implementation.
}
-------------------------------------------------
                               Single Paxos Entity
-------------------------------------------------

SinglePaxosEntity::SinglePaxosEntity(StableStorage::SingleEntry & my_entry, MessageReceiver & mr, Message msg)
{
	entry = my_entry;
	messageReceiver = mr;
	instanceId = msg.instanceId;
	
	clientMessage = 0;
	status = IDLE;
	
	if( msg is a message from client)
	{
		// That means we're the leader
		clientMessage = msg;
		sendPrepare();
	}
	else
	{
		// That means we are not leader, but a typical consensus member
		receiveMessage(msg)
	}
}

SinglePaxosEntity::receiveMessage(Message msg)
{
	// On EVERY message after success, we respond with a succeed message
	// This makes sync possible and faster after we've chosen the value
	if(entry.getSucceded())
	{
		Message succMsg;
		succMsg.messageType = SUCCEED;
		succMsg.instanceId = msg.instanceId;
		succMsg.sender = myProcessID;
		succMsg.data.value = ss.getValue();
		succMsg.size = sizeof(ss.getValue());
		mr.send(msg.sender, succMsg);
	}
	
	lastMessageTime = time(0);
	
	switch(msg.type)
	{
		case ALIVE:      // Used only for initralization of new instnaces
				break;   // for which we got no message yet
		case NACK:
				nackReceived(msg);
				break;
		case PREPARE:
				prepare(msg);
				break;
		case PROMISE:
				promise(msg);
				break;
		case VOTE:
				vote(msg);
				break;
		case VOTED:
				voted(msg);
				break;
		case SUCCEED:
				succeed(mgs);
				break;
		default:
				// some other message which is not handled now
	}
}

SinglePaxosEntity::check(bool amILeader)
{
	if(!amILeader)
	{
		// if i am not the leader, then set status to IDLE. It makes all previous 
		// state not active (like tryig or pooling).
		status = IDLE;
		
		// This means, if nearly for sure the consensus finished, and we don't know this
		if(time(0)-lastMessageTime > 5*FaliureDetector::timeout)
		{
			Message inquiry;
			inquiry.messageType = PREPARE;
			inquiry.instanceId = instanceId;
			inquiry.sender = myProcessID;
			// We cast a prepare for voting no. 0 - a  special no-voting val
			inquiry.data.value = 0;
			inquiry.data.size = sizeof(inquiry.data.value);
			
			mr.send(failureDetector.getLowestAliveId(), inquiry);
		}
	}
	else
	{
		switch(status)
		{
			case IDLE:
				// if we just become a leader, then start prepare phase
				sendPrepare();
				
				break;
			case TRYING:
				// if the prepare phase is too long, then repeat it
				if(time(0) - lastTryingTime > FaliureDetector::timeout)
					sendPrepare();
					
				break;
			case POOLING:
				// if voting phace is too long, then repeat everything from the begining
				if(time(0) - lastPoolingTime > FaliureDetector::timeout)
					sendPrepare();
					
				break;
		}
	}
}


SinglePaxosEntity::nackReceived(Message & msg)
{
	if(msg.ballotNumber != getLastTriedBallot()) return; // Not for this ballot
	if(msg.data.value != status) return; // Not for this phase
	
	switch(status) // We change the times to reinitiate proper round
	{
		case POOLING:
					lastPoolingTime=0;
		case TRYING:
					lastTryingTime=0;
	}
	
	check(true); // We force the instance to check our state, what means
	             // (because of the changed time) restarting some phase
}

SinglePaxosEntity::sendPrepare()
{
	status = TRYING;
	lastTryingTime = time(0);
	
	// we are starting new phase, so we increase the ballot number we start
	entry.setLastTriedBallot(entry.getLastTriedBallot() + 1);
	
	// start prepare phase
	Message prepareMsg;
	prepareMsg.messageType = PREPARE;
	prepareMsg.sender = myProcessID;
	prepareMsg.instanceId = msg.instanceId;
	prepareMsg.data.value = entry.lastTriedBallot();
	prepareMsg.data.value+sizeof(entry.lastTriedBallot)=entry.getValue();
	prepareMsg.size=sizeof(entry.lastTriedBallot)+sizeof(entry.getValue());
	
	messageReceiver.broadcastMessage(prepareMsg);
}

SinglePaxosEntity::sendVote()
{
	// Local atomicity requierd, so that no two local threads
	// will receive messages and get go to this part
	status = POLLING;   
	voters.clear();
	lastPoolingTime = time(0);
	
	if(valuesSentByPromisers.isEmpty())
	{
		// try to force new value from the client
		if( clientMessage == 0 )
		{
			entry.setValue(NOOP);
		}
		else
		{
			entry.setValue( Data(clientMessage.value, clientMessage.sender) );
		}
	}
	else
	{
		// because we receive some value from other proces
		// we have to force it
		
		// We're sorting the values from promisers, the latter as
		// most significant
		// I _do_ know that set must have no order, so in really
		// implementation it will be replaces by something having order.
		sort(valuesSentByPromisers, key, desc);
		// We extract the value from highest voting
		entry.setValue( valuesSentByPromisers->first().value );
	}
	
	// start voting phase
	Message voteMes;
	voteMes.messageType = VOTE;
	voteMes.sender = myProcessID;
	voteMes.instanceId = msg.instanceId;
	voteMes.data.value = entry.lastTriedBallot();
	voteMes.data.value+sizeof(entry.lastTriedBallot)=entry.getValue();
	voteMes.size=sizeof(entry.lastTriedBallot)+sizeof(entry.getValue());
	
	messageReceiver.broadcastMessage(voteMes);
}

SinglePaxosEntity::prepare(Message msg)
{
	// We eliminate the special value, used for getting val from lost consensus
	if(msg.ballotNumber == 0) return;
	
	if(msg.ballotNumber > entry.getLastPromise())
	{
		entry.setLastPromise(msg.ballotNumber);
		lastPromise = msg.ballotNumber;
		
		// response to the leader that we promising to take
		// part in this ballot (we didn't promise to any
		// ballot with higher number yet)
		Message response;
		response.type = PROMISE;
		response.instanceId = instanceId;
		response.sender = myProcessID;
		// If we have any value, we need to append it and its ballot number
		// to the message
		response.data.value=msg.ballotNumber;
		
		response.data.value+sizeof(ballotNumber) = entry.getLastVotedBallot();
		
		response.data.value+sizeof(ballotNumber)+
			sizeof(entry.getLastVotedBallot()) = entry.getValue(); 
		
		response.size = sizeof(ballotNumber)+
			sizeof(entry.getLastVotedBallot())+sizeof(entry.getValue());

		messageReceiver.sendMessage(msg.sender, response);
	}
	else
	{
		// Here we may response that there is ballot with higher
		// voting number
		Message nack;
		nack.messageType = NACK;
		nack.instanceId = msg.instanceId;
		nack.ballotNumber = entry.getLastPromise();
		nack.data.value = POOLING;
		nack.size = sizeof(nack.data.value);
		messageReceiver.sendMessage(msg.sender, nack);
	}
}

SinglePaxosEntity::promise(Message msg)
{
	if(status == TRYING and msg.data.value == entry.getLastTriedBallot())
	{
		// if sender is currently in promisers set then no action is taken,
		// otherwise senders is added to set of promisers
		promisers.union(msg.sender);
		lastTryingTime = time(0);
		
		// We're adding the values given by promisers to some set 
		valuesSentByPromisers.union(pair<msg.data+sizeof(vote_t), msg.data+2*sizeof(vote_t)>);
		
		// if we get promise message from majority of processes, then we can 
		// start sending vote messages
		if(promisers.size() > gm.getMajority())
		{                 
			sendVote();
		}		
	}
}

SinglePaxosEntity::vote(Message msg)
{
	if(entry.getLastPromise() == msg.data.value and msg.data.value > entry.getLastVotedBallot())
	{
		entry.setLastVotedBallot(msg.ballotNumber);
		entry.setValue(msg.data);
		
		// response that we are voting on this ballot
		Message response;
		response.messageType=VOTED;
		response.instanceId=msg.instanceId;
		response.sender=myProcessID;
		response.data.value=msg.ballotNumber;
		response.size=sizeof(msg.ballotNumber);
		
		mr.sendMessage(msg.sender, response);
	}
	else
	{
		// We can send some NACK to inform the leader
		// we already have higher voting number
		Message nack;
		nack.messageType = NACK;
		nack.instanceId = msg.instanceId;
		nack.ballotNumber = entry.getLastPromise();
		nack.data.value = TRYING;
		nack.size = sizeof(nack.data.value);
		messageReceiver.sendMessage(msg.sender, nack);
	}
}

SinglePaxosEntity::voted(Message msg)
{
	if(status == POLLING and msg.data.value == entry.getLastVotedBallot())
	{
		// if sender is currently in voters set then no action is taken,
		// otherwise senders is added to set of voters.
		voters.union(sender);
		lastPoolingTime = time(0);
		
		// if we get the voted message from majority of processes, then we can
		// end by sending succeed message to all processes
		if(voters.size() >= majority)
		{

			Message succMsg;
			succMsg.messageType = SUCCEED;
			succMsg.instanceId = msg.instanceId;
			succMsg.sender = myProcessID;
			succMsg.data.value = ss.getValue();
			succMsg.size = sizeof(ss.getValue());
			
			mr.broadcastMessage(succMsg); //(also to itself)
		}
	}
}

SinglePaxosEntity::succeed(Message msg)
{
	// the order of this operations are very important. If we swap them,
	// it will be possible that we set success value to true, having old
	// value for this instance. After crash we will read that this instance
	// was succeeded, but value will be incorrect.
	setValue(msg.data);
	setSucceeded(true);
}
			   
			   
-------------------------------------------------
                                  State Machine
-------------------------------------------------			   

StateMachine::StateMachine(MessageReceiver & mr)
{
	messageReceiver = mr;
}
			   
StateMachine::addValue(Data & data, int instanceId)
{
	// we can add this data to the list of all finished 
	// consensus data.
	commands[instanceId] = data;
	
	// try to expand the coherent part of commands
	while(commands.containsKey(lastCoherentPart + 1))
	{
		// this to lines should be executed atomic
		execute(commands[lastCoherentPart + 1]);
		lastCoherentPart++;
	}
}

StateMachine::execute(Data data)
{
	// here some action which this state machine do
	print(data.value);
	
	// send respone to client - does everyone send, or only the leader?
	// What if he crashes?
	
	// if(I am the leader)
	// {
		messageReceiver.send(data.klient, ACK);
	// }
}
			   
-------------------------------------------------
                                  Group Members
-------------------------------------------------

GroupMembers::GroupMembers(FailureDetector & fd)
{
	failureDetector = fd;
	
	// group must be formed HERE.
	
	// 1st way:
	// TODO: send a message to join the group
	
	// 2nd way:
	// Read config file
	
	const string groupFile;
	
	int number = 0;
	while ( ! end_of_file(groupFile) )
	{
		currentMembers.add( readFile(groupFile,number) );
		number++;
	}
	
	foreach(proc_id member, currentMembers)
	{
		// We set all the times for now, otherwise we would think we're the leader
		failureDetector.lastMessageReceived[member] = time(0);
	}
	
	// Optional: waiting a timeout for messages to know the group much better at
	// the beginning of consensus.
}

GroupMembers::isMember(proc_id)
{
	return currentMembers.exists(proc_id)
}

GroupMembers::addMember(proc_id id)
{
	// add member to list and notify failure detector
	// that group has changed
	currentMemebers.add(id);
	failureDetector.addMember(id);
}

GroupMembers::removeMember(proc_id id)
{
	// remove member from current members list and notify 
	// failure detector that current group has changed
	currentMembers.remove(id);
	failureDetector.removeMember(id);
}

GroupMembers::getMemberCount()
{
	return currentMembers.size();
}

GroupMembers::getMajority()
{
	return currentMembers.size() / 2 + 1;
}

-------------------------------------------------
                            Stable storage (hard disk)
-------------------------------------------------

StableStorage::SingleEntry::singleEntry(uint num, int & offset)
{
	lastOffset = offset; // warning: both are references!
	number = num; // Entry learns instance number it's offset in main file
	succeeded = false;
	lastBallot = lastPromise = NoVote;
}

StableStorage::SingleEntry::setSucceded(bool s)
{
	writeToFile(mainFile, number*sizeof(singleEntry), s);
	succeeded = s;
}

StableStorage::SingleEntry::setLastBallot(vote_t lb)
{
	writeToFile(mainFile, number*sizeof(singleEntry)+sizeof(succeeded), lb);
	lastBallot=lb;
}

StableStorage::SingleEntry::setLastPromise(vote_t)
{
	writeToFile(mainFile,
			number*sizeof(singleEntry) + sizeof(succeeded) + sizeof(lastBallot),
			lb);
	lastBallot=lb;
}

StableStorage::SingleEntry::setValue(value_t val)
{
	// Two operations here must be done without any other operations on offset
	// value; if one object is not enough for that, one lock should do.
	uint offset = lastOffset;
	lastOffset = lastOffset + val.size;

	writeToFile(valuesFile, offset, val.data.value);

	writeToFile(valuesFile,
				number*sizeof(singleEntry) + sizeof(succeeded) +
				sizeof(lastBallot) + sizeof(lastPromise),
				offset);
	offsetToValue = offset;

	writeToFile(valuesFile,
				number*sizeof(singleEntry) + sizeof(succeeded) +
				sizeof(lastBallot)+sizeof(lastPromise)+sizeof(offsetToValue),
				val.size);
	sizeOfValue = val.size;
	// Only if above one succeeds, the program considers value to be written.


}

StableStorage::SingleEntry::getValue()
{
	if(sizeOfValue == 0) return NULL; // that means a no-value (it's size is 0)
	return readFile(valuesFile,offsetToValue);
}

StableStorage::addNewEntry()
{
	paxosData.add( singleEntry(paxosData.count(), lastOffset) );
	writeToFile(stateFile,0,paxosData.count());
}

StableStorage::StableStorage()
{
	uint count = readFile(stateFile,0, sizeof(uint));

	addNewEntries(count);

	for(count times)
	{
		succeeded = readFile(mainFile, i*sizeof(singleEntry));
		lastPromise = readFile(mainFile, i*sizeof(singleEntry)+...);
		lastBallot = readFile(mainFile, i*sizeof(singleEntry)+...);
		offsetToValue = readFile(mainFile, i*sizeof(singleEntry)+...);
		sizeOfValue = readFile(mainFile, i*sizeof(singleEntry)+...);
	}
}


-------------------------------------------------
                                    Heartbeat
-------------------------------------------------

Heartbeat::Heartbeat(MessageReceiver & mr, groupMembers & gm)
{
	messageReceiver = mr;
	currentMembers = gm.currentMembers;
	
	// Preparing the 'alive' message.
	mes.sender = myProcessID;
	mes.messageType = ALIVE
	mes.instanceId = size = data = NULL; // We don't include these fields here
}

Heartbeat::heartBeater::run()
{
	while(isRunning)
	{
		// Broadcasting the message to others
		mr.broadcastMessage(mes);
		
		// The time _must_be double, otherwise most languages will have 0 in here!
		suspend thread for (double)( FaliureDetector::timeout / countPerTimeout );
	}
}

-------------------------------------------------
                                   Initialization
-------------------------------------------------
// this is example initialization of one multiPaxos instance.
void init()
{
	FailureDetector fd = FailureDetector();
	GroupMembers gm = GroupMembers(fd);
	MessageReceiver msgReceiver = MessageReceiver(fd, gm);
	StableStorage storage = StableStorage();
	MultiPaxos paxos = MultiPaxos(fd, msgReceiver, gm, storage);
}

\end{lstlisting}
