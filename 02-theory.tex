\chapter{Theory}

%Rozdział teoretyczny --- przegląd literatury naświetlający stan wiedzy na dany temat. 
%
%Przegląd literatury naświetlający stan wiedzy na dany temat obejmuje rozdziały pisane na podstawie
%literatury, której wykaz zamieszczany jest w części pracy pt.~\emph{Literatura} (lub inaczej \emph{Bibliografia},
%\emph{Piśmiennictwo}). W tekście pracy muszą wystąpić odwołania do wszystkich pozycji zamieszczonych w
%wykazie literatury. \textbf{Nie należy odnośników do literatury umieszczać w stopce strony.} Student jest
%bezwzględnie zobowiązany do wskazywania źródeł pochodzenia informacji przedstawianych w pracy,
%dotyczy to również rysunków, tabel, fragmentów kodu źródłowego programów itd. Należy także podać
%adresy stron internetowych w przypadku źródeł pochodzących z Internetu.

In this chapter the background for paxos algorithm is presented. First the consensus problem is described, th



\section{Definitions}

In order to prevent misunderstandng and to clarify the subject of the thesis we open it with definitions of
basic % TODO lepszy odpwoednik 
terms we use.


\paragraph{Consensus}
is a problem in distributed computing that encapsulates the task of group agreement in the presence of faluts.

\begin{description}
    \item[Validity] any value decided is a value proposed by some process,
    \item[Agreement] no two processes decide differently,
    \item[Termination] every correct process eventually decides,
    \item[Integrity] no process decides twice
\end{description}

\paragraph{State machine}
is any program, algorithm or protocol that can be described by it's state and that can transit to other state only by receiving a command.
There are no restrictions how the state may change.

\paragraph{Deterministic state machine}
is a state machine that from the same state under the same command will always change state in the same way.

Thanks to this property one may describe the state machine's state by the initial state and consecutive commands. 

\paragraph{Service}
is a program that receives requests -- or commands -- and execues them generating a response.

\paragraph{Deterministic service}
is a service, that will respond from given state and given command always with the same response, and will always change the state to the same state.

\paragraph{Client}
is the program sending requests to the service

\paragraph{Atomic -- or total order -- broadcast}
is a networking pritive providing send-to-all communication for which holds:
\begin{itemize}
 \item a message reaches all targets
 \item each two messages are delivered in the same order at every receiver
\end{itemize}

\paragraph{Failure}
is permanent lack od activity from a program. It may be caused by programming error, lack of electricity etc.

We do not consider byzantine crashes, i.e. a process may not misbehave in any way.

\paragraph{Crash model}
defines how the crashed processes may behave.
\begin{list}{}{ \setlength{\leftmargin}{0.2\textwidth} \setlength{\itemindent}{-0.1\textwidth}}
 \item[\textbf{Crash-Stop}] means that if a process failed, it failed permanently and will never be up again
 \item[\textbf{Crash-Recovery}] assumes that a crashed process may recover (i.e. start working again)
\end{list}


!! Poprawić definicje, sprawdzić poprawność !!


nierozwiazywalnosc paxos'a





