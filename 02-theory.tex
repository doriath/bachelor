\chapter{Theory}

%Rozdział teoretyczny --- przegląd literatury naświetlający stan wiedzy na dany temat. 
%
%Przegląd literatury naświetlający stan wiedzy na dany temat obejmuje rozdziały pisane na podstawie
%literatury, której wykaz zamieszczany jest w części pracy pt.~\emph{Literatura} (lub inaczej \emph{Bibliografia},
%\emph{Piśmiennictwo}). W tekście pracy muszą wystąpić odwołania do wszystkich pozycji zamieszczonych w
%wykazie literatury. \textbf{Nie należy odnośników do literatury umieszczać w stopce strony.} Student jest
%bezwzględnie zobowiązany do wskazywania źródeł pochodzenia informacji przedstawianych w pracy,
%dotyczy to również rysunków, tabel, fragmentów kodu źródłowego programów itd. Należy także podać
%adresy stron internetowych w przypadku źródeł pochodzących z Internetu.

In this chapter the background for paxos algorithm is presented.

Co to consensus, snapshot, service, catchup, deterministic state machine, state machine, 
crash models, atomic broadcast, nierozwiazywalnosc paxos'a

\section{Consensus}

Consensus is a problem in distributed computing that encapsulates the task of group agreement in the presence of faults.

\begin{description}
    \item[Validity] any value decided is a value proposed by some process,
    \item[Agreement] no two processes decide differently,
    \item[Termination] every correct process eventually decides,
    \item[Integrity] no process decides twice
\end{description}

The consensus problem is not solvable in asynchronous system where at least one process may crash and processes communicates by sending messages. This fact has been proved in FLP impossibility proof \cite{FLP}. 

This problem is related with atomic broadcast. If one can be solved, then the other also can be solved.

\section{State machine}

TODO: Definition, properties, etc.

\section{Fault models}

Crash stop

Crash recovery
