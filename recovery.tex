\chapter{Recovery}

In this chapter we describe the recovery of process from crash. We present 3 different algorithms implemented in JPaxos with different performance and fault tolerance. We will start from short overview of recovery and then present 4 different algorithms - crash-stop, crash-recovery with stable storage, epoch based recovery and view based recovery. After that we will summarize and compare all algorithms.

\section{Overview}

The crash of the process is a permanent lack of activity from a program. It may be caused by programming error, lack of electricity etc. Such process also losses all information not saved to stable storage.

So what is stable storage? The process can write data to different places, it can be processor cache, RAM, hard drives, etc. We can divide them into two categories - the one where data can survive the crash, and the other where data is lost after a crash. All memory holders from first category will be called stable storage and from second category non stable storage.

Usually operating system buffers the data before writing it to stable storages such as disks. Such writes are called asynchronous. They do not provide guarantee that if crash occurs, all data will be on the storage instead of being lost with the rest of system buffer.

Another option is to force the operating system to write the data to stable storage. Such operating is called \emph{synchronising} the buffers with storage. If we cal write and synchronise, we can say that the write is synchronous -- the data won't be lost due to crash. But writing to stable storage synchronously is very slow, so we want to minimize the amount of writes on critical path.

In following sections, we will use below variables: 
\begin{tightList}
  \item[$n$] -- number of processes
  \item[$f$] -- number of faulty process
  \item[$p$] -- the id of local process
\end{tightList}

\section{Crash stop}
\label{sec:crash_stop}

Let's start from the easiest scenario -- the crash-stop model. In this model crashed process will never be up again.
Because of that there is no recovery phase and we don't need a stable storage. Because process does not do any synchronous writes to stable storage the performance is very high. 

But there is one big drawback. In previous chapters we said that Paxos algorithm can decide on the value (can continue working) if majority of processes are correct. If majority of processes will crash in this model, the algorithm will never decide again and our replicated service will stop forever. Because of that using this model is not very practical and below we present 3 algorithms which allows for process recovering.

For performance tests and for theoretical purposes this model is very important -- it is the fastest model, used as comparison point with others. It is easier also to analyze correctness of this model and prove that the algorithms implementing recovery models are correct taking under consideration only the changes in the code.

\section{Recovery phase}

A recovery phase is the state of JPaxos replica after starting the process. It must be decided, if this is the first run of this replica or recovery from crash.

Of course, if this is the first start, the process can just go to normal state. Otherwise the process must stay in the Recovery phase as long as it does not fulfill requirements of the recovery model.

The process cannot join the Paxos protocol until the recovery phase is finished. So recovering process cannot respond to any message received, like \propose, \accept, \prepare and \prepareOK. However, these messages may be received and processed, so the process can passively join the protocol. The moment when the process joins Paxos protocol is also a moment when the process is considered as correct.


\section{Crash-recovery with stable storage}
\label{sec:full_ss}

First algorithm with ability to recover a process is crash recovery with stable storage. In this algorithm, we want to save all necessary data to stable storage. After crash the process can load all information from stable storage and join the Paxos protocol. To increase the performance the process should write as little as possible and as rare as possible to stable storage.

Description of variables used in algorithm below:
\begin{tightList}[\setlength{\labelwidth}{20em} \setlength{\leftmargin}{2\leftmargin}]
  \item[$log_p$] -- the array of consecutive instances -- $id \rightarrow <view, value>$
  \item[$view_p$] -- the current view number
  \item[$state_p$] -- the last snapshot made by service or received from catch-up
\end{tightList}

\begin{algorithmic}[1]
  \INIT{}
    \STATE $log_p \leftarrow \bot$ %\COMMENT {In-memory log}
    \STATE $view_p \leftarrow 0$
    \IF{$view_p \mod n = p$}
      \STATE $view_p \leftarrow 1$
    \ENDIF
    \STATE $state_p \leftarrow \bot$ %\COMMENT{The last snapshot}
    \STATE
    \IF{recovery after crash}
      \STATE $view_p \leftarrow$ last view number written to stable storage
      \FOR{$<id, view, value>$ from stable storage}
        \STATE $log_p[id] \leftarrow <view, value>$
      \ENDFOR
      \STATE $state_p \leftarrow$ last snapshot saved to stable storage
    \ENDIF
    \STATE
    \STATE join Paxos protocol
  \ENDINIT

  \vspace{1em}
  \PROCEDURE{advanceView(view)}
    \STATE write $view$ to stable storage
    \STATE $view_p \leftarrow view$
  \ENDPROC

  \vspace{1em}
  \PROCEDURE{updateValue(id, view, value)}
    \STATE $log_p[id] \leftarrow <view, value>$
    \STATE write $<id, view, value>$ to stable storage
  \ENDPROC

  \vspace{1em}
  \PROCEDURE{newSnapshot(snapshot)}
    \STATE write $snapshot$ to stable storage
    \STATE $state_p \leftarrow snapshot$ 
  \ENDPROC
\end{algorithmic}

On recovery the algorithm reads the view, values for all consensus instances and last snapshot from stable storage and can join a Paxos protocol (the process cannot respond to any message until it loads everything from stable storage).

This synchronous writes are made on critical path, so the performance of this algorithm will be much lower comparing to crash-stop model. But the crash recovery with stable storage algorithm tolerates catastrophic failures - the failure when all processes crashed. Of course when all processes are crashed, the replicated service will be unavailable. But we can start all processes again and they will recover to state from before the crash.

As we can see, this algorithm is already deployable in environment with crashes. But because the performance is really low, we will introduce two more algorithms. These algorithms will not tolerate catastrophic failures but their performance should be close to performance of crash-stop model.

\section{Epoch based recovery}
\label{sec:epoch_ss}

In this section, we will describe the algorithm based on epoch numbers. This algorithm is making only one synchronous write to stable storage on process startup, but the recovery phase is more complicated comparing to previous algorithm.

We will use following variables in the pseudocode:
\begin{tightList}[\setlength{\labelwidth}{20em} \setlength{\leftmargin}{3\leftmargin}]
  \item[$epoch_p$] -- vector of epoch numbers
  \item[$e$] -- epoch number of single process
  \item[$highestId$] -- the id of highest known consensus instance
\end{tightList}

The pseudocode below should explain this algorithm:
\begin{algorithmic}[1]
  \INIT{}
    \STATE $log_p \leftarrow \bot$ %\COMMENT {In-memory log}
    \STATE $view_p \leftarrow 0$
    \IF{$view_p \mod n = p$}
      \STATE $view_p \leftarrow view_p + 1$
    \ENDIF
    \STATE $state_p \leftarrow \bot$ %\COMMENT{The last snapshot}
    \STATE $\forall q : epoch_p[q] \leftarrow 0$
    \STATE
    \IF{recovery after crash}
      \STATE $epoch_p[p] \leftarrow$ last epoch number written to stable storage
      \STATE $epoch_p[p] \leftarrow epoch_p[p] + 1$
      \STATE send $<\recovery, epoch_p[p]>$ to all except $p$
      \STATE wait for $n-f <\recoveryAnswer, epoch, view, highestId>$ messages including from primary of highest view received
      \STATE $\forall s \in \Pi : epoch_p[s] \leftarrow \max\{{epoch[s] \; \mathrm{ received}}\}$
      \STATE $view_p \leftarrow \max\{{ view \; \mathrm{received}}\}$
      \STATE download all instance up to $highestId$ from leader using catch-up module
    \ENDIF
      \STATE write $epoch_p[p]$ to stable storage
    \STATE
    \STATE join Paxos protocol
  \ENDINIT

  \vspace{1em}
  \UPON{receive $<\recovery, e>$ from $q$}
    \IF{$q$ is primary of $view_p$}
      \STATE change to a higher view where $q$ is not primary
      \STATE wait until view change is complete
    \ENDIF
    \STATE $epoch_p[q] \leftarrow e$
    \STATE send $<\recoveryAnswer, epoch_p, view_p, highestId_p>$ to $q$
  \ENDUPON

  \vspace{1em}
  \UPON{$<\prepareOK, epoch, view, highestId>$ from $q$}
    \IF{$view > view_p$}
      \STATE $view_p \leftarrow view$
      \STATE send $<\prepare, view_p>$ to all
    \ELSE
      \STATE $\forall q : epoch_p[q] \leftarrow \max\{{epoch[q], epoch_q[q]}\}$
      \FORALL{$s \in \Pi$}
        \STATE discard all messages $<\prepareOK, epoch, view_p, -, ->$ from $s$ where $epoch[s] < epoch_q[p]$
      \ENDFOR
      \STATE execute normal handler 
    \ENDIF
  \ENDUPON
\end{algorithmic}

Epoch based recovery algorithms tolerates $f = \left\lfloor \frac{n-1}{2} \right\rfloor $ crashed processes. It is required that at least majority of processes are always correct - otherwise the algorithm will not be able to continue.

In recovery phase the process first notifies the majority of other replicas about it's recovery. As the process must know when it's state will be correct, it must wait for recovery answer from the leader, who has the most recent state. Once it gets the acknowledges, it is downloading all information from other replicas using catch-up mechanism.  When all necessary information is transfered, the replica can join the Paxos protocol and is considered as correct.

On normal execution this algorithm should be as fast as crash-stop model. The size of \prepareOK message is increased by epoch vector which contains $n$ numbers but it shouldn't have big impact on performance or network usage. This algorithm is a compromise between crash-stop and crash-recovery with stable storage.  

\subsection{Example}

Let's analyze one example to better understand this algorithm. Assume that we replicate service on $n = 5$ replicas. All processes are numbered from 0 to 4, have initial epoch vector $[0, 0, 0, 0, 0]$, are initially correct and are in view 4 (where process 4 is a leader).

Process 0 is suspecting a leader and changing view to 5 by sending \prepare message to all. Process 1 receives \prepare and respond with $<$\prepareOK, [0, 0, 0, 0, 0], 5, -$>$. The process 0 receives the response and process 1 crashes. After crash process 1 is recovering, increases epoch number to 1 and sends $<$\recovery, 1$>$ message to all. Processes [2, 3, 4] receives \recovery message, updates epoch vector to [0, 1, 0, 0, 0] and responds with \recoveryAnswer and view 4. After receiving three \recoveryAnswer messages with view 4 (also from leader of view 4), process 1 joins the protocol with view 4. Now process 2 receives the \prepare from process 0 and responds with $<$\prepareOK, [0, 1, 0, 0, 0], 5, -$>$.

Process 0 already received 3 \prepareOK responses, from [0, 1, 2]. But message from process 1 will be discarded because it was sent when process 1 was in epoch 0 and now from process 2 we know that process 1 is in epoch 1. If we wouldn't discard this message, process 0 would be prepared leader in view 5 but processes [1, 3, 4] would be in view 4 and that would cause violation of Paxos protocol.

\begin{TODO}
TODO dodac rysunek przedstawiajacy przyklad
\end{TODO}

\section{View based recovery}
\label{sec:view_ss}

The next algorithm is view based recovery. It has very similar properties as epoch based recovery. In every moment majority of processes is required to be correct but instead of using epoch numbers, the view number is written to table storage on every change. 

The pseudocode of this algorithm is presented below: 

\begin{algorithmic}[1]
  \INIT{}
    \STATE $log_p \leftarrow \bot$
    \STATE $view_p \leftarrow 0$
    \IF{$view_p \mod n = p$}
      \STATE $view_p \leftarrow view_p + 1$
    \ENDIF
    \STATE $state_p \leftarrow \bot$
    \STATE
    \IF{recovery after crash}
      \STATE $view_p \leftarrow$ last view number written to stable storage
      \IF{$view_p \mod n = p$}
        \STATE $view_p \leftarrow view_p + 1$
      \ENDIF
      \STATE send $<\recovery, view_p>$ to all except $p$
      \STATE wait for $n-f <\recoveryAnswer, view, highestId>$ messages including from primary of highest view received
      \STATE $view_p \leftarrow \max\{{ view \; \mathrm{received}}\}$
      \STATE download all instance up to $highestId$ from leader using catch-up module
    \ENDIF
    \STATE
    \STATE join Paxos protocol
  \ENDINIT

  \vspace{1em}
  \PROCEDURE{advanceView($view$)}
    \STATE write $view$ to stable storage
    \STATE $view_p \leftarrow view$
  \ENDPROC

  \vspace{1em}
  \UPON{receive $<\recovery, view>$ from $q$}
    \IF{$q$ is primary of $view_p$}
      \STATE change to a higher view where $q$ is not primary
      \STATE wait until view change is complete
    \ENDIF
    \IF{$view > view_p$}
      \STATE advanceView(view)
    \ENDIF
    \STATE send $<\recoveryAnswer, view_p, highestId_p>$ to $q$
  \ENDUPON
\end{algorithmic}

This algorithm makes synchronous write to stable storage on every view change. If the network is stable and the leader doesn't crash, performance of the algorithm should be the same as in crash-stop model. If the view is change often because of message loss or leader crash, the performance can decrease.

The recovery phase is quite similar to the epoch based recovery. Recovering process is sending \recovery message to all and waits for majority of \recoveryAnswer to get $heighestId$ value from leader. Then catch-up mechanism is used to retrieve missing instances from correct processes. This algorithm is easier than based on epoch, because it doesn't introduce any changes in handling \prepareOK message.

\section{Conclusions}

\begin{TODO}
Crash stop
\end{TODO}

Advantages:
\begin{itemize}
  \item high performance
  \item no synchronous writes to stable storage on critical path
\end{itemize}

Disadvantages:
\begin{itemize}
  \item tolerates only $f < \frac{n}{2}$ faulty processes
  \item faulty process cannot recovery 
  \item not very practical
\end{itemize}

Crash recovery with stable storage
Advantages:
\begin{itemize}
  \item tolerates catastrophic failures - $f = n$ 
\end{itemize}

Disadvantages:
\begin{itemize}
  \item synchronous writes to stable storage on critical path
  \item low performance
\end{itemize}


Proofs of all above algorithms can be found in TODO nuno draft.
\begin{TODO}
  reference to Nuno draft PaxosRecovery
\end{TODO}
