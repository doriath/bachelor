\chapter{Conclusions}

In this thesis, we have studied the problem of state machine replication based
on the Paxos algorithm defined in \cite{Lam98}. We described challenges faced in
the implementation phase to make the library ready to use in production systems,
with support of crash-recovery model, message loss and communication delays.

We first presented the Paxos algorithm with theoretical background. Then we
focused on the problems that occur when a single page of distributed algorithm pseudocode 
is transformed to ten thousand lines of production code. We presented
the architecture of our library, with short description of every module and data
structure used. The library has to handle unreliable network with message
loss and delays, so we presented the catch-up mechanism, which guarantees that
eventually every correct replica will be up to date. By introducing snapshot
mechanism, we solve the problem of keeping the log size bounded. To improve the efficiency of JPaxos, we implemented several performance improvements like concurrent
instances, skipping redundant messages and batching. To tolerate crash-recovery
model, four different recovery algorithms have been implemented and described in
this paper.

To test the library we implemented replicated hash map as an example of
deterministic service. We used it to verify that our library is working
correctly and that API we provide is user friendly. Our library is used as 
communication layer in PaxosSTM master thesis \cite{Tad10}.

Our future work includes introducing further performance improvements to JPaxos and
adding support for crash-recovery models without the stable storage at all. It~would be also interesting to evaluate all recovery algorithms in a network with
message loss and environment when failure of replica is a common scenario.
Finally, we would like to publish the library as open source project and test it
in production environment.

\section*{Acknowledgments}

% TODO fix the acknowledments. It is just copy-paste
The work described in chapters (...) was partially done by the authors during
their internships at Ecole Polytechnique Federal de Lausanne (EPFL) in July and
August 2009, under supervision of dr Nuno Santos and prof. Andre Schiper. We
thank prof. Schiper for mentoring us during our stay at EPFL. We thank Nuno for
[see the current version of diploma - they know better what to write here].

This work has been partially supported by the Polish Ministry of Science and
Higher Education within the European Regional Development Fund, Grant No.
POIG.01.03.01-00-008/08.  The two months stay at EPFL during the summer of 2009
was supported by EPFL within the internship program.

\noindent We would like to thank Nuno Santos from École Polytechnique Fédérale
de Lausanne for discussion about design of JPaxos, for making benchmarks, for
presenting recovery algorithms and for help with implementation issues. 

We also
would like to thank Tadeusz Kobus and Maciej Kokociński from Poznań University
of Technology, the developers of PaxosSTM, for testing our library and
discussion about JPaxos API.

\cleardoublepage
