\chapter{Conclusions}

%Zakończenie pracy zwane również Uwagami końcowymi lub Podsumowaniem powinno zawierać ustosunkowanie
%się autora do zadań wskazanych we wstępie do pracy, a w szczególności do celu i zakresu pracy oraz
%porównanie ich z faktycznymi wynikami pracy. Podejście takie umożliwia jasne określenie stopnia
%realizacji założonych celów oraz zwrócenie uwagi na wyniki osiągnięte przez autora w ramach jego
%samodzielnej pracy.
%
%Integralną częścią pracy są również dodatki, aneksy i załączniki np.~płyty CDROM
%zawierające stworzone w ramach pracy programy, aplikacje i projekty.

In this thesis, we have studied the problem of state machine replication based
on Paxos algorithm defined in \cite{Lam98}. We described challenges faced in
implementation phase to make the library ready to use in production systems
with support of crash-recovery model, message loss and communication delays.

We first presented the Paxos algorithm with theoretical background. Then we
focused on problems which occurs when transforming one page pseudocode of
distributed algorithm to few thousands line of production code. We presented
the architecture of our library with short description of every module and data
structures used. The library has to handle unreliable network with message
loss and delays, so we presented the catch-up mechanism which guarantees that
eventually every correct replica will be up to date. By introducing snapshot
mechanism, we solve a problem with infinite grow of log. To make the library
very efficient, we implemented several performance improvements like concurrent
instances, skipping redundant message and batching. To tolerate crash-recovery
model, four different recovery algorithms have been implemented and described in
this paper.

To test the library we implemented replicated hash map as an example of
deterministic service. We used it to verify that our library is working
correctly and that API we provide is user friendly. Our library is used as 
communication layer in PaxosSTM master thesis \cite{Tad10}.

Our future work includes further performance improvements to Paxos protocol and
implementing more crash-recovery algorithms with no use of stable storage.  It
would be also interesting to evaluate all recovery algorithm in network with
message loss and environment when failure of replica is a common scenario.
Finally we would like to publish the library as open source project and test it
in production environment.

% TODO JK Acknowledgments (Wojciechowski, Nuno?, Andre?, Tadek i Maciej?)
